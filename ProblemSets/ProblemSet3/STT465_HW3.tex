\documentclass[]{article}
\usepackage{lmodern}
\usepackage{amssymb,amsmath}
\usepackage{ifxetex,ifluatex}
\usepackage{fixltx2e} % provides \textsubscript
\ifnum 0\ifxetex 1\fi\ifluatex 1\fi=0 % if pdftex
  \usepackage[T1]{fontenc}
  \usepackage[utf8]{inputenc}
\else % if luatex or xelatex
  \ifxetex
    \usepackage{mathspec}
  \else
    \usepackage{fontspec}
  \fi
  \defaultfontfeatures{Ligatures=TeX,Scale=MatchLowercase}
\fi
% use upquote if available, for straight quotes in verbatim environments
\IfFileExists{upquote.sty}{\usepackage{upquote}}{}
% use microtype if available
\IfFileExists{microtype.sty}{%
\usepackage{microtype}
\UseMicrotypeSet[protrusion]{basicmath} % disable protrusion for tt fonts
}{}
\usepackage[margin=1in]{geometry}
\usepackage{hyperref}
\hypersetup{unicode=true,
            pdftitle={STT465\_HW3},
            pdfauthor={Sam Isken},
            pdfborder={0 0 0},
            breaklinks=true}
\urlstyle{same}  % don't use monospace font for urls
\usepackage{color}
\usepackage{fancyvrb}
\newcommand{\VerbBar}{|}
\newcommand{\VERB}{\Verb[commandchars=\\\{\}]}
\DefineVerbatimEnvironment{Highlighting}{Verbatim}{commandchars=\\\{\}}
% Add ',fontsize=\small' for more characters per line
\usepackage{framed}
\definecolor{shadecolor}{RGB}{248,248,248}
\newenvironment{Shaded}{\begin{snugshade}}{\end{snugshade}}
\newcommand{\KeywordTok}[1]{\textcolor[rgb]{0.13,0.29,0.53}{\textbf{#1}}}
\newcommand{\DataTypeTok}[1]{\textcolor[rgb]{0.13,0.29,0.53}{#1}}
\newcommand{\DecValTok}[1]{\textcolor[rgb]{0.00,0.00,0.81}{#1}}
\newcommand{\BaseNTok}[1]{\textcolor[rgb]{0.00,0.00,0.81}{#1}}
\newcommand{\FloatTok}[1]{\textcolor[rgb]{0.00,0.00,0.81}{#1}}
\newcommand{\ConstantTok}[1]{\textcolor[rgb]{0.00,0.00,0.00}{#1}}
\newcommand{\CharTok}[1]{\textcolor[rgb]{0.31,0.60,0.02}{#1}}
\newcommand{\SpecialCharTok}[1]{\textcolor[rgb]{0.00,0.00,0.00}{#1}}
\newcommand{\StringTok}[1]{\textcolor[rgb]{0.31,0.60,0.02}{#1}}
\newcommand{\VerbatimStringTok}[1]{\textcolor[rgb]{0.31,0.60,0.02}{#1}}
\newcommand{\SpecialStringTok}[1]{\textcolor[rgb]{0.31,0.60,0.02}{#1}}
\newcommand{\ImportTok}[1]{#1}
\newcommand{\CommentTok}[1]{\textcolor[rgb]{0.56,0.35,0.01}{\textit{#1}}}
\newcommand{\DocumentationTok}[1]{\textcolor[rgb]{0.56,0.35,0.01}{\textbf{\textit{#1}}}}
\newcommand{\AnnotationTok}[1]{\textcolor[rgb]{0.56,0.35,0.01}{\textbf{\textit{#1}}}}
\newcommand{\CommentVarTok}[1]{\textcolor[rgb]{0.56,0.35,0.01}{\textbf{\textit{#1}}}}
\newcommand{\OtherTok}[1]{\textcolor[rgb]{0.56,0.35,0.01}{#1}}
\newcommand{\FunctionTok}[1]{\textcolor[rgb]{0.00,0.00,0.00}{#1}}
\newcommand{\VariableTok}[1]{\textcolor[rgb]{0.00,0.00,0.00}{#1}}
\newcommand{\ControlFlowTok}[1]{\textcolor[rgb]{0.13,0.29,0.53}{\textbf{#1}}}
\newcommand{\OperatorTok}[1]{\textcolor[rgb]{0.81,0.36,0.00}{\textbf{#1}}}
\newcommand{\BuiltInTok}[1]{#1}
\newcommand{\ExtensionTok}[1]{#1}
\newcommand{\PreprocessorTok}[1]{\textcolor[rgb]{0.56,0.35,0.01}{\textit{#1}}}
\newcommand{\AttributeTok}[1]{\textcolor[rgb]{0.77,0.63,0.00}{#1}}
\newcommand{\RegionMarkerTok}[1]{#1}
\newcommand{\InformationTok}[1]{\textcolor[rgb]{0.56,0.35,0.01}{\textbf{\textit{#1}}}}
\newcommand{\WarningTok}[1]{\textcolor[rgb]{0.56,0.35,0.01}{\textbf{\textit{#1}}}}
\newcommand{\AlertTok}[1]{\textcolor[rgb]{0.94,0.16,0.16}{#1}}
\newcommand{\ErrorTok}[1]{\textcolor[rgb]{0.64,0.00,0.00}{\textbf{#1}}}
\newcommand{\NormalTok}[1]{#1}
\usepackage{graphicx,grffile}
\makeatletter
\def\maxwidth{\ifdim\Gin@nat@width>\linewidth\linewidth\else\Gin@nat@width\fi}
\def\maxheight{\ifdim\Gin@nat@height>\textheight\textheight\else\Gin@nat@height\fi}
\makeatother
% Scale images if necessary, so that they will not overflow the page
% margins by default, and it is still possible to overwrite the defaults
% using explicit options in \includegraphics[width, height, ...]{}
\setkeys{Gin}{width=\maxwidth,height=\maxheight,keepaspectratio}
\IfFileExists{parskip.sty}{%
\usepackage{parskip}
}{% else
\setlength{\parindent}{0pt}
\setlength{\parskip}{6pt plus 2pt minus 1pt}
}
\setlength{\emergencystretch}{3em}  % prevent overfull lines
\providecommand{\tightlist}{%
  \setlength{\itemsep}{0pt}\setlength{\parskip}{0pt}}
\setcounter{secnumdepth}{0}
% Redefines (sub)paragraphs to behave more like sections
\ifx\paragraph\undefined\else
\let\oldparagraph\paragraph
\renewcommand{\paragraph}[1]{\oldparagraph{#1}\mbox{}}
\fi
\ifx\subparagraph\undefined\else
\let\oldsubparagraph\subparagraph
\renewcommand{\subparagraph}[1]{\oldsubparagraph{#1}\mbox{}}
\fi

%%% Use protect on footnotes to avoid problems with footnotes in titles
\let\rmarkdownfootnote\footnote%
\def\footnote{\protect\rmarkdownfootnote}

%%% Change title format to be more compact
\usepackage{titling}

% Create subtitle command for use in maketitle
\newcommand{\subtitle}[1]{
  \posttitle{
    \begin{center}\large#1\end{center}
    }
}

\setlength{\droptitle}{-2em}

  \title{STT465\_HW3}
    \pretitle{\vspace{\droptitle}\centering\huge}
  \posttitle{\par}
    \author{Sam Isken}
    \preauthor{\centering\large\emph}
  \postauthor{\par}
      \predate{\centering\large\emph}
  \postdate{\par}
    \date{October 16, 2019}


\begin{document}
\maketitle

STT 465 – Fall 2019 Homework 3 – Due 10/23/2019 (In Class)
Instruction:

When using R in any problem, copy the code and results onto your word
document under that question number and add any required comments. You
will lose points if I do not see your codes. You should present a
stapled document when multiple pages are used. The grader will not be
held responsible for any loss of pages.

The crab data set (in D2L) contains information on the Crab width in cm
per female crab. After downloading onto your computer, import the data
into R using:

\begin{Shaded}
\begin{Highlighting}[]
\NormalTok{crab <-}\StringTok{ }\KeywordTok{read.table}\NormalTok{(}\StringTok{"crab.txt"}\NormalTok{,}\DataTypeTok{header=}\OtherTok{TRUE}\NormalTok{)}
\KeywordTok{head}\NormalTok{(crab)}
\end{Highlighting}
\end{Shaded}

\begin{verbatim}
##   color spine width nSatellites weight
## 1     3     3  28.3           8   3050
## 2     4     3  22.5           0   1550
## 3     2     1  26.0           9   2300
## 4     4     3  24.8           0   2100
## 5     4     3  26.0           4   2600
## 6     3     3  23.8           0   2100
\end{verbatim}

\begin{enumerate}
\def\labelenumi{\arabic{enumi}.}
\tightlist
\item
  Maximum Likelihood Inference of the mean in a Gaussian Model
\end{enumerate}

Use the carapace width variable to answer the following questions:

\begin{enumerate}
\def\labelenumi{\alph{enumi})}
\tightlist
\item
  Provide summary statistics (mean, variance, range, and histogram).
\end{enumerate}

\begin{Shaded}
\begin{Highlighting}[]
\NormalTok{mean_list <-}\StringTok{ }\KeywordTok{lapply}\NormalTok{(crab, mean)}
\NormalTok{variance_list <-}\StringTok{ }\KeywordTok{lapply}\NormalTok{(crab,var)}
\NormalTok{range_list <-}\StringTok{ }\KeywordTok{lapply}\NormalTok{(crab,range)}
\NormalTok{list_all <-}\StringTok{ }\KeywordTok{list}\NormalTok{(mean_list,variance_list,range_list)}
\NormalTok{result <-}\StringTok{ }\KeywordTok{setNames}\NormalTok{(}\KeywordTok{do.call}\NormalTok{(rbind.data.frame, list_all), }\KeywordTok{c}\NormalTok{(}\StringTok{"color"}\NormalTok{, }\StringTok{"spine"}\NormalTok{,}\StringTok{"width"}\NormalTok{,}\StringTok{"nSatellites"}\NormalTok{,}\StringTok{"weight"}\NormalTok{))}
\NormalTok{result}\OperatorTok{$}\NormalTok{Category <-}\StringTok{ }\KeywordTok{c}\NormalTok{(}\StringTok{"Means"}\NormalTok{,}\StringTok{"Variances"}\NormalTok{,}\StringTok{"Range_lower"}\NormalTok{,}\StringTok{"Range_upper"}\NormalTok{)}
\NormalTok{result}
\end{Highlighting}
\end{Shaded}

\begin{verbatim}
##        color    spine     width nSatellites     weight    Category
## 2  3.4393064 2.485549 26.298844    2.919075   2437.191       Means
## 21 0.6430972 0.681476  4.448138    9.912018 332958.097   Variances
## 3  2.0000000 1.000000 21.000000    0.000000   1200.000 Range_lower
## 4  5.0000000 3.000000 33.500000   15.000000   5200.000 Range_upper
\end{verbatim}

\begin{Shaded}
\begin{Highlighting}[]
\KeywordTok{library}\NormalTok{(tidyr)}
\KeywordTok{library}\NormalTok{(ggplot2)}

\NormalTok{crab }\OperatorTok\StringTok{ }\KeywordTok{gather}\NormalTok{() }\OperatorTok\StringTok{ }\KeywordTok{head}\NormalTok{()}
\end{Highlighting}
\end{Shaded}

\begin{verbatim}
##     key value
## 1 color     3
## 2 color     4
## 3 color     2
## 4 color     4
## 5 color     4
## 6 color     3
\end{verbatim}

\begin{Shaded}
\begin{Highlighting}[]
\KeywordTok{ggplot}\NormalTok{(}\KeywordTok{gather}\NormalTok{(crab), }\KeywordTok{aes}\NormalTok{(value)) }\OperatorTok{+}\StringTok{ }
\StringTok{    }\KeywordTok{geom_histogram}\NormalTok{(}\DataTypeTok{bins =} \DecValTok{10}\NormalTok{) }\OperatorTok{+}\StringTok{ }
\StringTok{    }\KeywordTok{facet_wrap}\NormalTok{(}\OperatorTok{~}\NormalTok{key, }\DataTypeTok{scales =} \StringTok{'free_x'}\NormalTok{)}
\end{Highlighting}
\end{Shaded}

\includegraphics{STT465_HW3_files/figure-latex/unnamed-chunk-3-1.pdf}

\begin{enumerate}
\def\labelenumi{\alph{enumi})}
\setcounter{enumi}{1}
\tightlist
\item
  Write the likelihood function under Gaussian assumptions.
\end{enumerate}

The Gaussian Assumptions attempt to estimate \(\mu , \sigma\) such that
our data is distributed as close to normally as possible. (Thus having a
pdf of:

\[ N(x)=\frac{1}{\sigma \sqrt{2 \pi}} e^{-\frac{(x-\mu)^{2}}{2 \sigma^{2}}} \]

and a plotted distribution of:

\begin{Shaded}
\begin{Highlighting}[]
\NormalTok{x <-}\StringTok{ }\KeywordTok{seq}\NormalTok{(}\OperatorTok{-}\DecValTok{4}\NormalTok{, }\DecValTok{4}\NormalTok{, }\DataTypeTok{length=}\DecValTok{100}\NormalTok{)}
\KeywordTok{plot}\NormalTok{(}\KeywordTok{dnorm}\NormalTok{(x))}
\end{Highlighting}
\end{Shaded}

\includegraphics{STT465_HW3_files/figure-latex/unnamed-chunk-4-1.pdf}

).

To do this we make 2 assumptions:

(mu and sigma must be known)

\begin{enumerate}
\def\labelenumi{\arabic{enumi}.}
\tightlist
\item
  Data must be independently distributed.
\item
  Data must be identically distributed.
\end{enumerate}

(This can be summarized as iid)

\[ f(x_1 , ...,x_n|\mu , \sigma) = \Pi \frac{1}{\sigma \sqrt{2\pi}} e^{\frac{-(x_i -\mu)^2}{2\sigma^2}} \]

\begin{enumerate}
\def\labelenumi{\alph{enumi})}
\setcounter{enumi}{2}
\tightlist
\item
  Derive the Maximum Likelihood estimator of the mean parameter (recall
  the steps:
\end{enumerate}

\begin{enumerate}
\def\labelenumi{\arabic{enumi}.}
\tightlist
\item
  write the likelihood,
\end{enumerate}

Let us first identify our likelihood function to estimate \(\hat{\mu}\)
assuming a Gaussian distribution

\[ f(x_1 , ...,x_n|\mu , \sigma) = \Pi \frac{1}{\sigma \sqrt{2\pi}} e^{\frac{-(x_i -\mu)^2}{2\sigma^2}} \]

\begin{enumerate}
\def\labelenumi{\arabic{enumi}.}
\setcounter{enumi}{1}
\tightlist
\item
  simplify as much as possible,
\end{enumerate}

We can simplify this to be:

\[ f(x_1 , ...,x_n|\mu , \sigma) = \frac{(2\pi)^{-n/2}}{\sigma^n} \text{exp} [-\frac{\Sigma(x_i-\mu)^2}{2\sigma^2}]  \]

\begin{enumerate}
\def\labelenumi{\arabic{enumi}.}
\setcounter{enumi}{2}
\tightlist
\item
  take the log, take derivative with respect to the mean,
\end{enumerate}

Now let us take the natural log of \(f(x_1 , ...,x_n|\mu , \sigma)\)

\[ \text{ln}(f(x_1 , ...,x_n|\mu , \sigma)) = \frac{1}{2}\text{n ln}(2\pi)-\text{n ln}(\sigma)-\frac{\Sigma(x_i-\mu)^2}{2\sigma^2}\]

Now let us take the derivitive with respect to \(\mu\) (because of this
many of the early terms drop out since they are constants)

\[ \frac{\delta(\text{ln}(f(x_1 , ...,x_n|\mu , \sigma)))}{\delta(\mu)} = \frac{\Sigma(x_i-\mu)^2}{2\sigma^2}  \]

\begin{enumerate}
\def\labelenumi{\arabic{enumi}.}
\setcounter{enumi}{3}
\tightlist
\item
  set the derivative equal to zero,
\end{enumerate}

\[ \frac{\Sigma(x_i-\mu)^2}{2\sigma^2} = 0  \]

\begin{enumerate}
\def\labelenumi{\arabic{enumi}.}
\setcounter{enumi}{4}
\tightlist
\item
  solve for the mean ).
\end{enumerate}

\[ \hat{\mu} = \frac{\Sigma x_i}{n}\]

This is expected as the MLE for a Gaussian distribution for \(\mu\) is
the sample mean.

\begin{enumerate}
\def\labelenumi{\alph{enumi})}
\setcounter{enumi}{3}
\tightlist
\item
  Provide the Max. Likelihood estimate of the mean for this data set
  together with an approximate 95\% CI (assume Central Limit Theorem).
\end{enumerate}

\begin{Shaded}
\begin{Highlighting}[]
\CommentTok{#Did for all variables }
\NormalTok{mle_list <-}\StringTok{ }\KeywordTok{lapply}\NormalTok{(crab, mean)}
\KeywordTok{print}\NormalTok{(mle_list)}
\end{Highlighting}
\end{Shaded}

\begin{verbatim}
## $color
## [1] 3.439306
## 
## $spine
## [1] 2.485549
## 
## $width
## [1] 26.29884
## 
## $nSatellites
## [1] 2.919075
## 
## $weight
## [1] 2437.191
\end{verbatim}

\begin{Shaded}
\begin{Highlighting}[]
\NormalTok{n <-}\StringTok{ }\KeywordTok{nrow}\NormalTok{(crab)}

\NormalTok{mle_color <-}\StringTok{ }\FloatTok{3.439306} \CommentTok{# xbar}
\NormalTok{mle_spine <-}\StringTok{ }\FloatTok{2.485549} \CommentTok{# xbar}
\NormalTok{mle_width <-}\StringTok{ }\FloatTok{26.29884} \CommentTok{# xbar}
\NormalTok{mle_nSatellites <-}\StringTok{ }\FloatTok{2.919075} \CommentTok{# xbar}
\NormalTok{mle_weight <-}\StringTok{ }\FloatTok{2437.191} \CommentTok{# xbar}

\NormalTok{sigma_color <-}\StringTok{ }\KeywordTok{sqrt}\NormalTok{(}\KeywordTok{var}\NormalTok{(crab}\OperatorTok{$}\NormalTok{color)) }\CommentTok{# sigma }
\NormalTok{sigma_spine <-}\StringTok{ }\KeywordTok{sqrt}\NormalTok{(}\KeywordTok{var}\NormalTok{(crab}\OperatorTok{$}\NormalTok{spine)) }\CommentTok{# sigma}
\NormalTok{sigma_width <-}\StringTok{ }\KeywordTok{sqrt}\NormalTok{(}\KeywordTok{var}\NormalTok{(crab}\OperatorTok{$}\NormalTok{width)) }\CommentTok{# sigma}
\NormalTok{sigma_nSatellites <-}\StringTok{ }\KeywordTok{sqrt}\NormalTok{(}\KeywordTok{var}\NormalTok{(crab}\OperatorTok{$}\NormalTok{nSatellites)) }\CommentTok{# sigma}
\NormalTok{sigma_weight <-}\StringTok{ }\KeywordTok{sqrt}\NormalTok{(}\KeywordTok{var}\NormalTok{(crab}\OperatorTok{$}\NormalTok{weight)) }\CommentTok{# sigma}

\CommentTok{# Confidence Intervals }
\NormalTok{color_CI <-}\StringTok{ }\KeywordTok{c}\NormalTok{(mle_color }\OperatorTok{-}\StringTok{ }\NormalTok{(}\FloatTok{1.96}\OperatorTok{*}\NormalTok{(sigma_color}\OperatorTok{/}\KeywordTok{sqrt}\NormalTok{(n))),mle_color }\OperatorTok{+}\StringTok{ }\NormalTok{(}\FloatTok{1.96}\OperatorTok{*}\NormalTok{(sigma_color}\OperatorTok{/}\KeywordTok{sqrt}\NormalTok{(n))))}
\NormalTok{spine_CI <-}\StringTok{ }\KeywordTok{c}\NormalTok{(mle_spine }\OperatorTok{-}\StringTok{ }\NormalTok{(}\FloatTok{1.96}\OperatorTok{*}\NormalTok{(sigma_spine}\OperatorTok{/}\KeywordTok{sqrt}\NormalTok{(n))),mle_spine }\OperatorTok{+}\StringTok{ }\NormalTok{(}\FloatTok{1.96}\OperatorTok{*}\NormalTok{(sigma_spine}\OperatorTok{/}\KeywordTok{sqrt}\NormalTok{(n))))}
\NormalTok{width_CI <-}\StringTok{ }\KeywordTok{c}\NormalTok{(mle_width }\OperatorTok{-}\StringTok{ }\NormalTok{(}\FloatTok{1.96}\OperatorTok{*}\NormalTok{(sigma_width}\OperatorTok{/}\KeywordTok{sqrt}\NormalTok{(n))),mle_width }\OperatorTok{+}\StringTok{ }\NormalTok{(}\FloatTok{1.96}\OperatorTok{*}\NormalTok{(sigma_width}\OperatorTok{/}\KeywordTok{sqrt}\NormalTok{(n))))}
\NormalTok{nSatellites_CI <-}\StringTok{ }\KeywordTok{c}\NormalTok{(mle_nSatellites }\OperatorTok{-}\StringTok{ }\NormalTok{(}\FloatTok{1.96}\OperatorTok{*}\NormalTok{(sigma_nSatellites}\OperatorTok{/}\KeywordTok{sqrt}\NormalTok{(n))),mle_nSatellites }\OperatorTok{+}\StringTok{ }\NormalTok{(}\FloatTok{1.96}\OperatorTok{*}\NormalTok{(sigma_nSatellites}\OperatorTok{/}\KeywordTok{sqrt}\NormalTok{(n))))}
\NormalTok{weight_CI <-}\StringTok{ }\KeywordTok{c}\NormalTok{(mle_weight }\OperatorTok{-}\StringTok{ }\NormalTok{(}\FloatTok{1.96}\OperatorTok{*}\NormalTok{(sigma_weight}\OperatorTok{/}\KeywordTok{sqrt}\NormalTok{(n))),mle_weight }\OperatorTok{+}\StringTok{ }\NormalTok{(}\FloatTok{1.96}\OperatorTok{*}\NormalTok{(sigma_weight}\OperatorTok{/}\KeywordTok{sqrt}\NormalTok{(n))))}
\end{Highlighting}
\end{Shaded}

\begin{enumerate}
\def\labelenumi{\arabic{enumi}.}
\setcounter{enumi}{1}
\tightlist
\item
  Bayesian Inference: Normal Model
\end{enumerate}

\begin{enumerate}
\def\labelenumi{\alph{enumi})}
\tightlist
\item
  Assume the normal model with inference for \(\mu\) where
  \(\sigma^{2}\) is known. Analyze the carapace width of crabs with the
  normal model with conjugate prior distribution for \(\mu\) in which
  \(\{ \mu_0 = 20, \sigma^2_0 = 1.2, \kappa_0 \}\) and
  \(\sigma^2_0 = \frac{\sigma^2}{\kappa_0}\) . Find the posterior
  distribution for \(\mu\).
\end{enumerate}

We know that:

\[ \kappa_0 = \frac{n}{n+\frac{\sigma^2}{\sigma^2_0}} \]

Let us find our n:

\begin{Shaded}
\begin{Highlighting}[]
\NormalTok{n <-}\StringTok{ }\KeywordTok{length}\NormalTok{(crab}\OperatorTok{$}\NormalTok{width)}
\NormalTok{n}
\end{Highlighting}
\end{Shaded}

\begin{verbatim}
## [1] 173
\end{verbatim}

thus, n = 173.

Let us now calculate our credibility factor \(\kappa_0\).

\[ \kappa_0 = \frac{173}{173 +\frac{\sigma^2}{1.2}} \]

Let us plug that into our formula for \(\sigma^2_0\) to solve for
\(\sigma^2\).

\begin{Shaded}
\begin{Highlighting}[]
\NormalTok{sigma_sq <-}\StringTok{ }\KeywordTok{var}\NormalTok{(crab}\OperatorTok{$}\NormalTok{width)}
\end{Highlighting}
\end{Shaded}

\[ \sigma^2_0 = \frac{\sigma^2}{\kappa_0} = \frac{\sigma^2}{\frac{173}{173 +\frac{\sigma^2}{1.2}}} = 1.2 \]

(Scratch work shown on attached sheet)

Thus we discover:

\[ \sigma^2 = 1.2 \] So

\[ \kappa_0 = \frac{173}{173 +\frac{1.2}{1.2}} = .99425 \]

Let us now calculate the posterior mean, median, mode and variance.

\[ \mu_* = (1-\kappa_0)\mu_0 + \kappa_0*\bar{Y} \]

where \$\bar\{Y\} \$ is

\begin{Shaded}
\begin{Highlighting}[]
\NormalTok{ybar <-}\StringTok{ }\KeywordTok{mean}\NormalTok{(crab}\OperatorTok{$}\NormalTok{width)}
\NormalTok{ybar}
\end{Highlighting}
\end{Shaded}

\begin{verbatim}
## [1] 26.29884
\end{verbatim}

thus

\[ \mu_* = (1-.99425)20 + .99425*26.29884 \]

\begin{Shaded}
\begin{Highlighting}[]
\NormalTok{mustar <-}\StringTok{ }\NormalTok{((}\DecValTok{1}\OperatorTok{-}\NormalTok{.}\DecValTok{99425}\NormalTok{)}\OperatorTok{*}\DecValTok{20}\NormalTok{) }\OperatorTok{+}\StringTok{ }\NormalTok{(.}\DecValTok{99425}\OperatorTok{*}\FloatTok{26.29884}\NormalTok{)}
\NormalTok{mustar}
\end{Highlighting}
\end{Shaded}

\begin{verbatim}
## [1] 26.26262
\end{verbatim}

Since the posterior distribution is normal, the posterior mean =
posterior median = posterior mode.

Let us now calculate the posterior variance:

\[ \sigma^2_* = \frac{\kappa_0 * \sigma^2}{n} \]

\begin{Shaded}
\begin{Highlighting}[]
\NormalTok{sigma_squared_star <-}\StringTok{ }\NormalTok{(.}\DecValTok{99425} \OperatorTok{*}\StringTok{ }\FloatTok{1.2}\NormalTok{)}\OperatorTok{/}\DecValTok{173}
\NormalTok{sigma_squared_star}
\end{Highlighting}
\end{Shaded}

\begin{verbatim}
## [1] 0.006896532
\end{verbatim}

Thus the posterior distribution of \(\mu\) is:

\[\mu|y \text{~} \text{N} (26.26262 \text{ , } 0.006896532)\]

\begin{enumerate}
\def\labelenumi{\alph{enumi})}
\setcounter{enumi}{1}
\tightlist
\item
  Construct a 90\% credibility interval for \(\mu\)𝜇 and interpret
  your result.
\end{enumerate}

\[\mu_* \pm Z_.05 \sqrt{\sigma^2_*} \]

\begin{Shaded}
\begin{Highlighting}[]
\KeywordTok{abs}\NormalTok{(}\KeywordTok{qnorm}\NormalTok{(.}\DecValTok{05}\NormalTok{,}\DataTypeTok{lower.tail=}\OtherTok{TRUE}\NormalTok{))}
\end{Highlighting}
\end{Shaded}

\begin{verbatim}
## [1] 1.644854
\end{verbatim}

Thus:

\[ 26.26262 \pm 1.644854(\sqrt{0.006896532}) \]

\begin{Shaded}
\begin{Highlighting}[]
\CommentTok{#Credibility Interval Calculated}
\NormalTok{lower_bound <-}\StringTok{ }\FloatTok{26.26262} \OperatorTok{-}\StringTok{ }\NormalTok{(}\FloatTok{1.644854}\OperatorTok{*}\NormalTok{(}\KeywordTok{sqrt}\NormalTok{(}\FloatTok{0.006896532}\NormalTok{)))}
\NormalTok{upper_bound <-}\StringTok{ }\FloatTok{26.26262} \OperatorTok{+}\StringTok{ }\NormalTok{(}\FloatTok{1.644854}\OperatorTok{*}\NormalTok{(}\KeywordTok{sqrt}\NormalTok{(}\FloatTok{0.006896532}\NormalTok{)))}
\NormalTok{lower_bound}
\end{Highlighting}
\end{Shaded}

\begin{verbatim}
## [1] 26.12602
\end{verbatim}

\begin{Shaded}
\begin{Highlighting}[]
\NormalTok{upper_bound}
\end{Highlighting}
\end{Shaded}

\begin{verbatim}
## [1] 26.39922
\end{verbatim}

\begin{enumerate}
\def\labelenumi{\alph{enumi})}
\setcounter{enumi}{2}
\tightlist
\item
  Plot the prior density, likelihood and posterior density on a single
  plot clearly indicating each graph.
\end{enumerate}

\begin{Shaded}
\begin{Highlighting}[]
\NormalTok{y =}\StringTok{ }\NormalTok{crab}\OperatorTok{$}\NormalTok{width }\CommentTok{# data}
\NormalTok{n =}\StringTok{ }\KeywordTok{length}\NormalTok{(y)}
\NormalTok{sigma2=}\StringTok{ }\FloatTok{1.2} \CommentTok{# known mean}

\NormalTok{#### Parameters for prior density (usually called hyperparameters)}
\NormalTok{a=}\DecValTok{3}
\NormalTok{b=}\DecValTok{2}

\NormalTok{a0=}\DecValTok{3} \CommentTok{# prior shape parameter}
\NormalTok{b0=}\DecValTok{2}  \CommentTok{# prior scale(rate) parameter}
\KeywordTok{options}\NormalTok{(}\DataTypeTok{digits=}\DecValTok{4}\NormalTok{)}
\NormalTok{a=a0}\OperatorTok{+}\NormalTok{n}\OperatorTok{/}\DecValTok{2}\NormalTok{; }
\KeywordTok{c}\NormalTok{(a,sigma2,b)}
\end{Highlighting}
\end{Shaded}

\begin{verbatim}
## [1] 89.5  1.2  2.0
\end{verbatim}

\begin{Shaded}
\begin{Highlighting}[]
\CommentTok{# Inference on sigma2 = 1/lambda }

\NormalTok{sig2priormean=b0}\OperatorTok{/}\NormalTok{(a0}\OperatorTok{-}\DecValTok{1}\NormalTok{)  }\CommentTok{# Prior mean for sigma^2}

\NormalTok{sig2postmean=b}\OperatorTok{/}\NormalTok{(a}\OperatorTok{-}\DecValTok{1}\NormalTok{)}

\NormalTok{### 95% Cred. Int}

\NormalTok{sig2cr=}\DecValTok{1}\OperatorTok{/}\KeywordTok{qgamma}\NormalTok{(}\KeywordTok{c}\NormalTok{(}\FloatTok{0.975}\NormalTok{,}\FloatTok{0.025}\NormalTok{),a,b)}

\KeywordTok{c}\NormalTok{(ybar, mustar, sig2cr)}
\end{Highlighting}
\end{Shaded}

\begin{verbatim}
## [1] 26.29884 26.26262  0.01835  0.02781
\end{verbatim}

\begin{Shaded}
\begin{Highlighting}[]
\NormalTok{sig2v=}\KeywordTok{seq}\NormalTok{(}\FloatTok{0.01}\NormalTok{,}\DecValTok{10}\NormalTok{,}\FloatTok{0.01}\NormalTok{) }\CommentTok{# HERE }
\NormalTok{prior=}\KeywordTok{dgamma}\NormalTok{(}\DecValTok{1}\OperatorTok{/}\NormalTok{sigma_squared_star,a0,b0)}\OperatorTok{/}\NormalTok{sigma_squared_star}\OperatorTok{^}\DecValTok{2}
\NormalTok{post=}\KeywordTok{dgamma}\NormalTok{(}\DecValTok{1}\OperatorTok{/}\NormalTok{sigma_squared_star,a,b)}\OperatorTok{/}\NormalTok{sigma_squared_star}\OperatorTok{^}\DecValTok{2}\NormalTok{;}
\NormalTok{like=}\KeywordTok{dgamma}\NormalTok{(}\DecValTok{1}\OperatorTok{/}\NormalTok{sigma_squared_star,a}\OperatorTok{-}\NormalTok{a0}\OperatorTok{-}\DecValTok{1}\NormalTok{,b}\OperatorTok{-}\NormalTok{b0}\OperatorTok{+}\DecValTok{0}\NormalTok{)}\OperatorTok{/}\NormalTok{sigma_squared_star}\OperatorTok{^}\DecValTok{2}
\KeywordTok{plot}\NormalTok{(}\KeywordTok{c}\NormalTok{(}\DecValTok{0}\NormalTok{,}\DecValTok{10}\NormalTok{),}\KeywordTok{c}\NormalTok{(}\DecValTok{0}\NormalTok{,}\FloatTok{1.2}\NormalTok{),}\DataTypeTok{type=}\StringTok{"n"}\NormalTok{,}
     \DataTypeTok{main=}\StringTok{"Inference on the model variance parameter"}\NormalTok{,}
     \DataTypeTok{xlab=}\StringTok{"sigma^2 = 1/lambda"}\NormalTok{,}\DataTypeTok{ylab=}\StringTok{"density/likelihood"}\NormalTok{)}
\KeywordTok{lines}\NormalTok{(sigma_squared_star,prior,}\DataTypeTok{lty=}\DecValTok{3}\NormalTok{,}\DataTypeTok{lwd=}\DecValTok{2}\NormalTok{); }\KeywordTok{lines}\NormalTok{(sigma_squared_star,like,}\DataTypeTok{lty=}\DecValTok{3}\NormalTok{,}\DataTypeTok{lwd=}\DecValTok{2}\NormalTok{)}
\KeywordTok{lines}\NormalTok{(sigma_squared_star,post,}\DataTypeTok{lty=}\DecValTok{3}\NormalTok{,}\DataTypeTok{lwd=}\DecValTok{2}\NormalTok{)}

\KeywordTok{points}\NormalTok{(sig2cr,}\KeywordTok{c}\NormalTok{(}\DecValTok{0}\NormalTok{,}\DecValTok{0}\NormalTok{),}\DataTypeTok{pch=}\KeywordTok{rep}\NormalTok{(}\DecValTok{16}\NormalTok{,}\DecValTok{2}\NormalTok{),}\DataTypeTok{cex=}\KeywordTok{rep}\NormalTok{(}\FloatTok{1.5}\NormalTok{,}\DecValTok{2}\NormalTok{))}
\KeywordTok{legend}\NormalTok{(}\FloatTok{1.8}\NormalTok{,}\FloatTok{1.2}\NormalTok{,}
       \KeywordTok{c}\NormalTok{(}\StringTok{"Prior density"}\NormalTok{,}\StringTok{"Likelihood function (normalised)"}\NormalTok{,}\StringTok{"Posterior density"}\NormalTok{),}
       \DataTypeTok{lty=}\KeywordTok{c}\NormalTok{(}\DecValTok{1}\NormalTok{,}\DecValTok{2}\NormalTok{,}\DecValTok{3}\NormalTok{),}\DataTypeTok{lwd=}\KeywordTok{c}\NormalTok{(}\DecValTok{2}\NormalTok{,}\DecValTok{2}\NormalTok{,}\DecValTok{2}\NormalTok{))}
\end{Highlighting}
\end{Shaded}

\includegraphics{STT465_HW3_files/figure-latex/unnamed-chunk-13-1.pdf}


\end{document}
